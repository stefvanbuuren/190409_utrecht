\documentclass[ignorenonframetext,aspectratio=43]{beamer}
\setbeamertemplate{caption}[numbered]
\setbeamertemplate{caption label separator}{: }
\setbeamercolor{caption name}{fg=normal text.fg}
\beamertemplatenavigationsymbolsempty
\usepackage{lmodern}
\usepackage{amssymb,amsmath}
\usepackage{ifxetex,ifluatex}
\usepackage{fixltx2e} % provides \textsubscript
\ifnum 0\ifxetex 1\fi\ifluatex 1\fi=0 % if pdftex
  \usepackage[T1]{fontenc}
  \usepackage[utf8]{inputenc}
\else % if luatex or xelatex
  \ifxetex
    \usepackage{mathspec}
  \else
    \usepackage{fontspec}
  \fi
  \defaultfontfeatures{Ligatures=TeX,Scale=MatchLowercase}
\fi
\usecolortheme{beaver}
% use upquote if available, for straight quotes in verbatim environments
\IfFileExists{upquote.sty}{\usepackage{upquote}}{}
% use microtype if available
\IfFileExists{microtype.sty}{%
\usepackage{microtype}
\UseMicrotypeSet[protrusion]{basicmath} % disable protrusion for tt fonts
}{}
\newif\ifbibliography
\hypersetup{
            pdftitle={Recipes for multilevel imputation},
            pdfauthor={Stef van Buuren (Utrecht University)},
            pdfborder={0 0 0},
            breaklinks=true}
\urlstyle{same}  % don't use monospace font for urls
\usepackage{color}
\usepackage{fancyvrb}
\newcommand{\VerbBar}{|}
\newcommand{\VERB}{\Verb[commandchars=\\\{\}]}
\DefineVerbatimEnvironment{Highlighting}{Verbatim}{commandchars=\\\{\}}
% Add ',fontsize=\small' for more characters per line
\usepackage{framed}
\definecolor{shadecolor}{RGB}{248,248,248}
\newenvironment{Shaded}{\begin{snugshade}}{\end{snugshade}}
\newcommand{\KeywordTok}[1]{\textcolor[rgb]{0.13,0.29,0.53}{\textbf{#1}}}
\newcommand{\DataTypeTok}[1]{\textcolor[rgb]{0.13,0.29,0.53}{#1}}
\newcommand{\DecValTok}[1]{\textcolor[rgb]{0.00,0.00,0.81}{#1}}
\newcommand{\BaseNTok}[1]{\textcolor[rgb]{0.00,0.00,0.81}{#1}}
\newcommand{\FloatTok}[1]{\textcolor[rgb]{0.00,0.00,0.81}{#1}}
\newcommand{\ConstantTok}[1]{\textcolor[rgb]{0.00,0.00,0.00}{#1}}
\newcommand{\CharTok}[1]{\textcolor[rgb]{0.31,0.60,0.02}{#1}}
\newcommand{\SpecialCharTok}[1]{\textcolor[rgb]{0.00,0.00,0.00}{#1}}
\newcommand{\StringTok}[1]{\textcolor[rgb]{0.31,0.60,0.02}{#1}}
\newcommand{\VerbatimStringTok}[1]{\textcolor[rgb]{0.31,0.60,0.02}{#1}}
\newcommand{\SpecialStringTok}[1]{\textcolor[rgb]{0.31,0.60,0.02}{#1}}
\newcommand{\ImportTok}[1]{#1}
\newcommand{\CommentTok}[1]{\textcolor[rgb]{0.56,0.35,0.01}{\textit{#1}}}
\newcommand{\DocumentationTok}[1]{\textcolor[rgb]{0.56,0.35,0.01}{\textbf{\textit{#1}}}}
\newcommand{\AnnotationTok}[1]{\textcolor[rgb]{0.56,0.35,0.01}{\textbf{\textit{#1}}}}
\newcommand{\CommentVarTok}[1]{\textcolor[rgb]{0.56,0.35,0.01}{\textbf{\textit{#1}}}}
\newcommand{\OtherTok}[1]{\textcolor[rgb]{0.56,0.35,0.01}{#1}}
\newcommand{\FunctionTok}[1]{\textcolor[rgb]{0.00,0.00,0.00}{#1}}
\newcommand{\VariableTok}[1]{\textcolor[rgb]{0.00,0.00,0.00}{#1}}
\newcommand{\ControlFlowTok}[1]{\textcolor[rgb]{0.13,0.29,0.53}{\textbf{#1}}}
\newcommand{\OperatorTok}[1]{\textcolor[rgb]{0.81,0.36,0.00}{\textbf{#1}}}
\newcommand{\BuiltInTok}[1]{#1}
\newcommand{\ExtensionTok}[1]{#1}
\newcommand{\PreprocessorTok}[1]{\textcolor[rgb]{0.56,0.35,0.01}{\textit{#1}}}
\newcommand{\AttributeTok}[1]{\textcolor[rgb]{0.77,0.63,0.00}{#1}}
\newcommand{\RegionMarkerTok}[1]{#1}
\newcommand{\InformationTok}[1]{\textcolor[rgb]{0.56,0.35,0.01}{\textbf{\textit{#1}}}}
\newcommand{\WarningTok}[1]{\textcolor[rgb]{0.56,0.35,0.01}{\textbf{\textit{#1}}}}
\newcommand{\AlertTok}[1]{\textcolor[rgb]{0.94,0.16,0.16}{#1}}
\newcommand{\ErrorTok}[1]{\textcolor[rgb]{0.64,0.00,0.00}{\textbf{#1}}}
\newcommand{\NormalTok}[1]{#1}
\usepackage{longtable,booktabs}
\usepackage{caption}
% These lines are needed to make table captions work with longtable:
\makeatletter
\def\fnum@table{\tablename~\thetable}
\makeatother
\usepackage{graphicx,grffile}
\makeatletter
\def\maxwidth{\ifdim\Gin@nat@width>\linewidth\linewidth\else\Gin@nat@width\fi}
\def\maxheight{\ifdim\Gin@nat@height>\textheight0.8\textheight\else\Gin@nat@height\fi}
\makeatother
% Scale images if necessary, so that they will not overflow the page
% margins by default, and it is still possible to overwrite the defaults
% using explicit options in \includegraphics[width, height, ...]{}
\setkeys{Gin}{width=\maxwidth,height=\maxheight,keepaspectratio}

% Prevent slide breaks in the middle of a paragraph:
\widowpenalties 1 10000
\raggedbottom

\AtBeginPart{
  \let\insertpartnumber\relax
  \let\partname\relax
  \frame{\partpage}
}
\AtBeginSection{
  \ifbibliography
  \else
    \let\insertsectionnumber\relax
    \let\sectionname\relax
    \frame{\sectionpage}
  \fi
}
\AtBeginSubsection{
  \let\insertsubsectionnumber\relax
  \let\subsectionname\relax
  \frame{\subsectionpage}
}

\setlength{\parindent}{0pt}
\setlength{\parskip}{6pt plus 2pt minus 1pt}
\setlength{\emergencystretch}{3em}  % prevent overfull lines
\providecommand{\tightlist}{%
  \setlength{\itemsep}{0pt}\setlength{\parskip}{0pt}}
\setcounter{secnumdepth}{0}
\setbeamertemplate{footline}[text line]{%
  \parbox{\linewidth}{\vspace*{-8pt}Recipes for multilevel imputation - 190409 Utrecht\hfill\insertshortauthor\hfill\insertpagenumber}}
\setbeamertemplate{navigation symbols}{}

\title{Recipes for multilevel imputation}
\author{Stef van Buuren (Utrecht University)}
\date{April 9, 2019}

\begin{document}
\frame{\titlepage}

\begin{frame}[fragile]{Main question}

\emph{Can we use \texttt{mice} for multilevel data, and if so, how?}

\end{frame}

\begin{frame}{Imputation by fully conditional specification}

\begin{center}\includegraphics{recipes_files/figure-beamer/unnamed-chunk-1-1} \end{center}

\end{frame}

\begin{frame}{Imputation by fully conditional specification}

\begin{center}\includegraphics{recipes_files/figure-beamer/unnamed-chunk-2-1} \end{center}

\end{frame}

\begin{frame}{Imputation by fully conditional specification}

\begin{center}\includegraphics{recipes_files/figure-beamer/unnamed-chunk-3-1} \end{center}

\end{frame}

\begin{frame}{Imputation by fully conditional specification}

\begin{center}\includegraphics{recipes_files/figure-beamer/unnamed-chunk-4-1} \end{center}

\end{frame}

\begin{frame}{Imputation by fully conditional specification}

\begin{center}\includegraphics{recipes_files/figure-beamer/unnamed-chunk-5-1} \end{center}

\end{frame}

\begin{frame}{Imputation by fully conditional specification - next
iteration}

\begin{center}\includegraphics{recipes_files/figure-beamer/unnamed-chunk-6-1} \end{center}

\end{frame}

\begin{frame}{Imputation by fully conditional specification - next
iteration}

\begin{center}\includegraphics{recipes_files/figure-beamer/unnamed-chunk-7-1} \end{center}

\end{frame}

\begin{frame}{\texttt{brandsma} data}

\begin{itemize}
\tightlist
\item
  Brandsma and Knuver, Int J Ed Res, 1989.
\item
  Extensively discussed in Snijders and Bosker (2012), 2nd ed.
\item
  4106 pupils, 216 schools, about 4\% missing values
\end{itemize}

\end{frame}

\begin{frame}[fragile]{\texttt{brandsma} data subset}

\begin{Shaded}
\begin{Highlighting}[]
\KeywordTok{library}\NormalTok{(mice)}
\NormalTok{d <-}\StringTok{ }\NormalTok{brandsma[, }\KeywordTok{c}\NormalTok{(}\StringTok{"sch"}\NormalTok{, }\StringTok{"lpo"}\NormalTok{, }\StringTok{"sex"}\NormalTok{, }\StringTok{"den"}\NormalTok{)]}
\KeywordTok{head}\NormalTok{(d, }\DecValTok{2}\NormalTok{)}
\end{Highlighting}
\end{Shaded}

\begin{verbatim}
##   sch lpo sex den
## 1   1  NA   1   1
## 2   1  50   1   1
\end{verbatim}

\begin{itemize}
\tightlist
\item
  \texttt{sch}: School number, cluster variable, \(C = 216\);
\item
  \texttt{lpo}: Language test post, outcome at pupil level;
\item
  \texttt{sex}: Sex of pupil, predictor at pupil level (0-1);
\item
  \texttt{den}: School denomination, predictor at school level (1-4).
\end{itemize}

\end{frame}

\begin{frame}[fragile]{Model of scientific interest}

Predict \texttt{lpo} from the

\begin{itemize}
\tightlist
\item
  level-1 predictor \texttt{sex}
\item
  level-2 predictor \texttt{den}
\end{itemize}

\end{frame}

\begin{frame}{Level notation - Bryk and Raudenbush (1992)}

\begin{align}
{{\texttt{lpo}}}_{ic} & = \beta_{0c} + \beta_{1c}{{\texttt{sex}}}_{ic} + \epsilon_{ic}\\
\beta_{0c}     & = \gamma_{00} + \gamma_{01}{{\texttt{den}}}_{c} + u_{0c}\\
\beta_{1c}     & = \gamma_{10}
\end{align}

\begin{itemize}
\tightlist
\item
  \(\text{lpo}_{ic}\) is the test score of pupil \(i\) in school \(c\)
\item
  \(\text{sex}_{ic}\) is the sex of pupil \(i\) in school \(c\)
\item
  \(\text{den}_c\) is the religious denomination of school \(c\)
\item
  \(\beta_{0c}\) is a random intercept that varies by cluster
\item
  \(\beta_{1c}\) is a sex effect, assumed to be the same across schools.
\item
  \(\epsilon_{ic} \sim N(0, \sigma_\epsilon^2)\) is the within-cluster
  random residual at the pupil level
\end{itemize}

\end{frame}

\begin{frame}{Where are the missings?}

In single level data, missingness may be in the outcome and/or in the
predictors

With multilevel data, missingness may be in:

\begin{enumerate}
\def\labelenumi{\arabic{enumi}.}
\item
  the outcome variable;
\item
  the level-1 predictors;
\item
  the level-2 predictors;
\item
  the class variable.
\end{enumerate}

\end{frame}

\begin{frame}{Univariate missing, level-1 outcome}

\begin{center}\includegraphics{recipes_files/figure-beamer/unnamed-chunk-9-1} \end{center}

\end{frame}

\begin{frame}{Univariate missing, level-1 predictor, sporadically
missing}

\begin{center}\includegraphics{recipes_files/figure-beamer/unnamed-chunk-10-1} \end{center}

\end{frame}

\begin{frame}{Univariate missing, level-1 predictor, systematically
missing}

\begin{center}\includegraphics{recipes_files/figure-beamer/unnamed-chunk-11-1} \end{center}

\end{frame}

\begin{frame}{Univariate missing, level-2 predictor}

\begin{center}\includegraphics{recipes_files/figure-beamer/unnamed-chunk-12-1} \end{center}

\end{frame}

\begin{frame}{Multivariate missing}

\begin{center}\includegraphics{recipes_files/figure-beamer/unnamed-chunk-13-1} \end{center}

\end{frame}

\begin{frame}{Fully conditional specification for multilevel data}

\begin{align}
\dot{{\texttt{lpo}}}_{ic} & \sim N(\beta_0 + \beta_1 {{\texttt{den}}}_{c} + \beta_2 {{\texttt{sex}}}_{ic} + u_{0c}, \sigma_\epsilon^2)\\
\dot{{\texttt{sex}}}_{ic} & \sim N(\beta_0 + \beta_1 {{\texttt{den}}}_{c} + \beta_2 {{\texttt{lpo}}}_{ic} + u_{0c}, \sigma_\epsilon^2)
\end{align}

\end{frame}

\begin{frame}{Theoretical problem with FCS}

Conditional expectation of \(\texttt{sex}_{ic}\) in a random effects
model depends on

\begin{itemize}
\tightlist
\item
  \(\texttt{lpo}_{ic}\),
\item
  \(\overline{\texttt{lpo}}_{i}\), the mean of cluster \(i\), and
\item
  \(n_i\), the size of cluster \(i\).
\end{itemize}

Resche-Rigon \& White (2018) suggest the imputation model

\begin{itemize}
\tightlist
\item
  should incorporate the cluster means of level-1 predictors
\item
  be heteroscedastic if cluster sizes vary
\end{itemize}

\end{frame}

\begin{frame}{General imputation/modeling sequence - START SIMPLE}

\begin{enumerate}
\def\labelenumi{\arabic{enumi}.}
\tightlist
\item
  Pick a simple complete-data model
\item
  Create imputations using an imputation template
\item
  Check the imputes (convergence/plausibility)
\item
  Estimate parameters
\item
  Make complete-data model more realistic, go to 1.
\end{enumerate}

See \url{https://stefvanbuuren.name/fimd/sec-mlguidelines.html}

\end{frame}

\begin{frame}{Seven imputation templates, increasing complexity}

\begin{enumerate}
\def\labelenumi{\arabic{enumi}.}
\tightlist
\item
  \emph{Intercept-only model, missing outcomes}
\item
  \emph{Random intercepts, missing level-1 predictor}
\item
  Random intercepts, contextual model
\item
  \emph{Random intercepts, missing level-2 predictor}
\item
  Random intercepts, interactions
\item
  Random slopes, missing outcomes and predictors
\item
  Random slopes, interactions
\end{enumerate}

\end{frame}

\begin{frame}{1 Intercept-only model, missing outcomes (model)}

\begin{align}
{{\texttt{lpo}}}_{ic} & = \beta_{0c} + \epsilon_{ic}\\
\beta_{0c}     & = \gamma_{00} + u_{0c}
\end{align}

\end{frame}

\begin{frame}[fragile]{1 Intercept-only model, missing outcomes
(imputation)}

\begin{Shaded}
\begin{Highlighting}[]
\NormalTok{d <-}\StringTok{ }\NormalTok{brandsma[, }\KeywordTok{c}\NormalTok{(}\StringTok{"sch"}\NormalTok{, }\StringTok{"lpo"}\NormalTok{)]}
\NormalTok{pred <-}\StringTok{ }\KeywordTok{make.predictorMatrix}\NormalTok{(d)}
\NormalTok{pred[}\StringTok{"lpo"}\NormalTok{, }\StringTok{"sch"}\NormalTok{] <-}\StringTok{ }\OperatorTok{-}\DecValTok{2}
\NormalTok{imp <-}\StringTok{ }\KeywordTok{mice}\NormalTok{(d, }\DataTypeTok{pred =}\NormalTok{ pred, }\DataTypeTok{meth =} \StringTok{"2l.pmm"}\NormalTok{, }\DataTypeTok{m =} \DecValTok{10}\NormalTok{, }\DataTypeTok{maxit =} \DecValTok{1}\NormalTok{,}
            \DataTypeTok{print =} \OtherTok{FALSE}\NormalTok{, }\DataTypeTok{seed =} \DecValTok{152}\NormalTok{)}
\end{Highlighting}
\end{Shaded}

\end{frame}

\begin{frame}[fragile]{1 Intercept-only model, missing outcomes
(analysis)}

\begin{Shaded}
\begin{Highlighting}[]
\KeywordTok{library}\NormalTok{(lme4)}
\end{Highlighting}
\end{Shaded}

\begin{verbatim}
## Loading required package: Matrix
\end{verbatim}

\begin{Shaded}
\begin{Highlighting}[]
\NormalTok{fit <-}\StringTok{ }\KeywordTok{with}\NormalTok{(imp, }\KeywordTok{lmer}\NormalTok{(lpo }\OperatorTok{~}\StringTok{ }\NormalTok{(}\DecValTok{1} \OperatorTok{|}\StringTok{ }\NormalTok{sch), }\DataTypeTok{REML =} \OtherTok{FALSE}\NormalTok{))}
\KeywordTok{summary}\NormalTok{(}\KeywordTok{pool}\NormalTok{(fit))}
\end{Highlighting}
\end{Shaded}

\begin{verbatim}
##             estimate std.error statistic   df p.value
## (Intercept)     40.9     0.322       127 3368       0
\end{verbatim}

\end{frame}

\begin{frame}[fragile]{1 Intercept-only model, missing outcomes
(variances)}

\begin{Shaded}
\begin{Highlighting}[]
\KeywordTok{library}\NormalTok{(mitml)}
\KeywordTok{testEstimates}\NormalTok{(}\KeywordTok{as.mitml.result}\NormalTok{(fit), }\DataTypeTok{var.comp =} \OtherTok{TRUE}\NormalTok{)}\OperatorTok{$}\NormalTok{var.comp}
\end{Highlighting}
\end{Shaded}

\begin{verbatim}
##                          Estimate
## Intercept~~Intercept|sch   18.021
## Residual~~Residual         63.306
## ICC|sch                     0.222
\end{verbatim}

\end{frame}

\begin{frame}[fragile]{2 Random intercepts, missing level-1 (model)}

\begin{align}
{{\texttt{lpo}}}_{ic} & = \beta_{0c} + \beta_{1c}{{\texttt{iqv}}}_{ic} + \epsilon_{ic}\\
\beta_{0c}     & = \gamma_{00} + u_{0c}\\
\beta_{1c}     & = \gamma_{10}
\end{align}

\begin{itemize}
\tightlist
\item
  Missing values in both \texttt{lpo} and \texttt{iqv}
\end{itemize}

\end{frame}

\begin{frame}[fragile]{2 Random intercepts, missing level-1
(imputation)}

\begin{Shaded}
\begin{Highlighting}[]
\NormalTok{d <-}\StringTok{ }\NormalTok{brandsma[, }\KeywordTok{c}\NormalTok{(}\StringTok{"sch"}\NormalTok{, }\StringTok{"lpo"}\NormalTok{, }\StringTok{"iqv"}\NormalTok{)]}
\NormalTok{pred <-}\StringTok{ }\KeywordTok{make.predictorMatrix}\NormalTok{(d)}
\NormalTok{pred[}\StringTok{"lpo"}\NormalTok{, ] <-}\StringTok{ }\KeywordTok{c}\NormalTok{(}\OperatorTok{-}\DecValTok{2}\NormalTok{, }\DecValTok{0}\NormalTok{, }\DecValTok{3}\NormalTok{)}
\NormalTok{pred[}\StringTok{"iqv"}\NormalTok{, ] <-}\StringTok{ }\KeywordTok{c}\NormalTok{(}\OperatorTok{-}\DecValTok{2}\NormalTok{, }\DecValTok{3}\NormalTok{, }\DecValTok{0}\NormalTok{)}
\NormalTok{imp <-}\StringTok{ }\KeywordTok{mice}\NormalTok{(d, }\DataTypeTok{pred =}\NormalTok{ pred, }\DataTypeTok{meth =} \StringTok{"2l.pmm"}\NormalTok{, }\DataTypeTok{seed =} \DecValTok{919}\NormalTok{,}
            \DataTypeTok{m =} \DecValTok{10}\NormalTok{, }\DataTypeTok{print =} \OtherTok{FALSE}\NormalTok{)}
\end{Highlighting}
\end{Shaded}

\begin{itemize}
\tightlist
\item
  Impute \texttt{lpo} from \texttt{iqv} \emph{and} the cluster means of
  \texttt{iqv}
\item
  Impute \texttt{iqv} from \texttt{lpo} \emph{and} the cluster means of
  \texttt{lpo}
\item
  Alternative: Use \texttt{mitml::panImpute()} or
  \texttt{mitml::jomoImpute()}
\end{itemize}

\end{frame}

\begin{frame}[fragile]{2 Random intercepts, missing level-1
(predictorMatrix)}

\begin{Shaded}
\begin{Highlighting}[]
\NormalTok{pred}
\end{Highlighting}
\end{Shaded}

\begin{verbatim}
##     sch lpo iqv
## sch   0   1   1
## lpo  -2   0   3
## iqv  -2   3   0
\end{verbatim}

\end{frame}

\begin{frame}[fragile]{2 Random intercepts, missing level-1 (analysis)}

\begin{Shaded}
\begin{Highlighting}[]
\NormalTok{fit <-}\StringTok{ }\KeywordTok{with}\NormalTok{(imp, }\KeywordTok{lmer}\NormalTok{(lpo }\OperatorTok{~}\StringTok{  }\NormalTok{iqv }\OperatorTok{+}\StringTok{ }\NormalTok{(}\DecValTok{1} \OperatorTok{|}\StringTok{ }\NormalTok{sch), }\DataTypeTok{REML =} \OtherTok{FALSE}\NormalTok{))}
\KeywordTok{summary}\NormalTok{(}\KeywordTok{pool}\NormalTok{(fit))}
\end{Highlighting}
\end{Shaded}

\begin{verbatim}
##             estimate std.error statistic   df p.value
## (Intercept)    40.96    0.2378       172 3337       0
## iqv             2.52    0.0525        48 2127       0
\end{verbatim}

\begin{Shaded}
\begin{Highlighting}[]
\KeywordTok{testEstimates}\NormalTok{(}\KeywordTok{as.mitml.result}\NormalTok{(fit), }\DataTypeTok{var.comp =} \OtherTok{TRUE}\NormalTok{)}\OperatorTok{$}\NormalTok{var.comp}
\end{Highlighting}
\end{Shaded}

\begin{verbatim}
##                          Estimate
## Intercept~~Intercept|sch    9.479
## Residual~~Residual         40.862
## ICC|sch                     0.188
\end{verbatim}

\end{frame}

\begin{frame}[fragile]{4 Random intercepts, missing level-2 predictor
(model)}

\begin{align}
{{\texttt{lpo}}}_{ic} & = \beta_{0c} + \beta_{1c}{{\texttt{iqv}}}_{ic} + \epsilon_{ic}\\
\beta_{0c}     & = \gamma_{00} + \gamma_{01}{{\texttt{den}}}_{c} + u_{0c}\\
\beta_{1c}     & = \gamma_{10}
\end{align}

\begin{itemize}
\tightlist
\item
  Missing values in \texttt{lpo}, \texttt{iqv} and \texttt{den}
\item
  For \texttt{den} the imputation model uses school level aggregates
\end{itemize}

\end{frame}

\begin{frame}[fragile]{4 Random intercepts, missing level-2
(imputation)}

\begin{Shaded}
\begin{Highlighting}[]
\NormalTok{d <-}\StringTok{ }\NormalTok{brandsma[, }\KeywordTok{c}\NormalTok{(}\StringTok{"sch"}\NormalTok{, }\StringTok{"lpo"}\NormalTok{, }\StringTok{"iqv"}\NormalTok{, }\StringTok{"den"}\NormalTok{)]}
\NormalTok{meth <-}\StringTok{ }\KeywordTok{make.method}\NormalTok{(d)}
\NormalTok{meth[}\KeywordTok{c}\NormalTok{(}\StringTok{"lpo"}\NormalTok{, }\StringTok{"iqv"}\NormalTok{, }\StringTok{"den"}\NormalTok{)] <-}\StringTok{ }\KeywordTok{c}\NormalTok{(}\StringTok{"2l.pmm"}\NormalTok{, }\StringTok{"2l.pmm"}\NormalTok{,}
                                  \StringTok{"2lonly.pmm"}\NormalTok{)}
\NormalTok{pred <-}\StringTok{ }\KeywordTok{make.predictorMatrix}\NormalTok{(d)}
\NormalTok{pred[}\StringTok{"lpo"}\NormalTok{, ] <-}\StringTok{ }\KeywordTok{c}\NormalTok{(}\OperatorTok{-}\DecValTok{2}\NormalTok{, }\DecValTok{0}\NormalTok{, }\DecValTok{3}\NormalTok{, }\DecValTok{1}\NormalTok{)}
\NormalTok{pred[}\StringTok{"iqv"}\NormalTok{, ] <-}\StringTok{ }\KeywordTok{c}\NormalTok{(}\OperatorTok{-}\DecValTok{2}\NormalTok{, }\DecValTok{3}\NormalTok{, }\DecValTok{0}\NormalTok{, }\DecValTok{1}\NormalTok{)}
\NormalTok{pred[}\StringTok{"den"}\NormalTok{, ] <-}\StringTok{ }\KeywordTok{c}\NormalTok{(}\OperatorTok{-}\DecValTok{2}\NormalTok{, }\DecValTok{1}\NormalTok{, }\DecValTok{1}\NormalTok{, }\DecValTok{0}\NormalTok{)}
\NormalTok{imp <-}\StringTok{ }\KeywordTok{mice}\NormalTok{(d, }\DataTypeTok{pred =}\NormalTok{ pred, }\DataTypeTok{meth =}\NormalTok{ meth, }\DataTypeTok{seed =} \DecValTok{418}\NormalTok{,}
            \DataTypeTok{m =} \DecValTok{10}\NormalTok{, }\DataTypeTok{print =} \OtherTok{FALSE}\NormalTok{)}
\end{Highlighting}
\end{Shaded}

\end{frame}

\begin{frame}[fragile]{4 Random intercepts, missing level-2
(predictorMatrix)}

\begin{Shaded}
\begin{Highlighting}[]
\NormalTok{pred}
\end{Highlighting}
\end{Shaded}

\begin{verbatim}
##     sch lpo iqv den
## sch   0   1   1   1
## lpo  -2   0   3   1
## iqv  -2   3   0   1
## den  -2   1   1   0
\end{verbatim}

\end{frame}

\begin{frame}{4 Random intercepts, missing level-2 (density)}

\includegraphics{recipes_files/figure-beamer/mladenspmm-1.pdf}

\end{frame}

\begin{frame}[fragile]{4 Random intercepts, missing level-2 (analysis)}

\begin{verbatim}
##                 estimate std.error statistic   df  p.value
## (Intercept)       40.071    0.4549     88.09  187 0.000000
## iqv                2.516    0.0532     47.34 1242 0.000000
## as.factor(den)2    2.041    0.5925      3.45  430 0.000589
## as.factor(den)3    0.234    0.6519      0.36  285 0.719226
## as.factor(den)4    1.843    1.1642      1.58 1041 0.113706
\end{verbatim}

\begin{verbatim}
##                          Estimate
## Intercept~~Intercept|sch    8.621
## Residual~~Residual         40.761
## ICC|sch                     0.175
\end{verbatim}

\end{frame}

\begin{frame}{Classic recipe for single-level data: Which predictors?}

\begin{enumerate}
\def\labelenumi{\arabic{enumi}.}
\tightlist
\item
  Include all variables that appear in the complete-data model
\item
  Include variables related to the nonresponse
\item
  Include variables that explain a considerable amount of variance
\item
  Remove from variables selected in steps 2 and 3 those variables that
  have too many missing values within the subgroup of incomplete cases
\end{enumerate}

\emph{Does this recipe also apply to multilevel data?}

\end{frame}

\begin{frame}[fragile]{Recipe: Missing level-1}

\begin{longtable}[]{@{}ll@{}}
\toprule
& Recipe for a level-1 target\tabularnewline
\midrule
\endhead
1. & Define the most general analytic model\tabularnewline
2. & Select a \texttt{2l} method that imputes close to the
data\tabularnewline
3. & Include all level-1 variables\tabularnewline
4. & Include the disaggregated cluster means of level-1
variables\tabularnewline
5. & Include all level-1 interactions implied by analytic
model\tabularnewline
6. & Include all level-2 predictors\tabularnewline
7. & Include all level-2 interactions implied by analytic
model\tabularnewline
8. & Include all cross-level interactions implied by analytic
model\tabularnewline
9. & Include predictors related to the missingness and the
target\tabularnewline
10. & Exclude any terms involving the target\tabularnewline
\bottomrule
\end{longtable}

\end{frame}

\begin{frame}[fragile]{Recipe: Missing level-2}

\begin{longtable}[]{@{}ll@{}}
\toprule
& Recipe for a level-2 target\tabularnewline
\midrule
\endhead
1. & Define the most general analytic model\tabularnewline
2. & Select a \texttt{2lonly} method that imputes close to the
data\tabularnewline
3. & Include the cluster means of all level-1 variables\tabularnewline
4. & Include the cluster means of all level-1
interactions\tabularnewline
5. & Include all level-2 predictors\tabularnewline
6. & Include all interactions of level-2 variables\tabularnewline
7. & Include predictors related to the missingness and
target\tabularnewline
8. & Exclude any terms involving the target\tabularnewline
\bottomrule
\end{longtable}

\end{frame}

\begin{frame}[fragile]{Conclusion}

\emph{Can we use \texttt{mice} for multilevel data, and if so, how?}

\begin{itemize}
\tightlist
\item
  Hot spot of current research
\item
  Multilevel imputation: more complex, but doable
\item
  Start simple, take small steps
\item
  Build upon templates and modeling recipes
\item
  Study \url{https://stefvanbuuren.name/fimd/sec-mlguidelines.html}
\item
  Gain confidence at each step
\item
  Start playing around\ldots{}
\end{itemize}

\end{frame}

\end{document}
